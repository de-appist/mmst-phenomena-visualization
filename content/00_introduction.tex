\section*{Introduction}

\subsection*{Task Description}

The fuel entering the furnace goes through a combustion reaction and a portion of the released heat 
(because of the heat loss) is transferred to the feed stream, increasing its temperature. In the
visualization of a furnace in this project, the following information should be conveyed to the user: 
\newline
1. There are three streams entering the furnace: the fuel, air, and the feed, and two outlet 
stream: the exhaust and the hot stream (like what we can see in the conventional view of a 
furnace) 
\newline
2. The fuel goes through a combustion reaction.
\newline
3. There is a heat loss; and thus, not all of the released heat is used for heating up the feed 
stream.
\newline
4. When the flow rate of the fuel to the furnace increases, and so does the released heat, the 
temperature of the outlet stream increment.
\newline
5. Increasing the feed flow rate results in a decrease in the temperature of the outlet stream. 

\subsection*{Introduction}

The human can often be the deciding factor if a process works correctly or not. It can be diffucult to  estimate the outcome certain interactions between human and machine cause. Therefore it is important for the Operator to understand the process he or she is working on.
\newline
This article focuses on visualizing the phenomena happening in a furnace, to give the user an understanding of the system and consequently reduce mistakes, potentially made by an operator.
\newline
The furnace, whom is looked at in this project, has 3 inputs (inlet, air and fuel) and 2 ouputs (outlet and combustion products). The stream gets heated by the energy released from the fuel. It is a simple model and serves as concept to show our tools used to make a process understandable.
\newline

\addcontentsline{toc}{section}{Introduction}
